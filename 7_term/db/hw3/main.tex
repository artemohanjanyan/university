\documentclass{article}

\usepackage{amsmath}
\usepackage{amssymb}
\usepackage{amsthm}
\usepackage{mathtext}
\usepackage{mathtools}
\usepackage[T1,T2A]{fontenc}
\usepackage[utf8]{inputenc}
\usepackage[left=2cm,right=2cm,top=2cm,bottom=2cm]{geometry}
\usepackage{microtype}
\usepackage{enumitem}
\usepackage{bm}
\usepackage{cancel}
\usepackage{proof}
\usepackage{epigraph}
\usepackage{titlesec}
\usepackage[dvipsnames]{xcolor}
\usepackage{stmaryrd}
\usepackage{cellspace}
\usepackage{cmll}
\usepackage{multirow}
\usepackage{booktabs}
\usepackage{tikz}
\usepackage{caption}
\usepackage{wrapfig}
\usepackage{minted}
\usepackage{svg}

\usepackage[hidelinks]{hyperref}

\usepackage[russian]{babel}
\selectlanguage{russian}

\hypersetup{%
    colorlinks=true,
    linkcolor=blue
}

\DeclareMathOperator{\StudentId}{StudentId}
\DeclareMathOperator{\StudentName}{StudentName}
\DeclareMathOperator{\GroupId}{GroupId}
\DeclareMathOperator{\GroupName}{GroupName}
\DeclareMathOperator{\CourseId}{CourseId}
\DeclareMathOperator{\CourseName}{CourseName}
\DeclareMathOperator{\LecturerId}{LecturerId}
\DeclareMathOperator{\LecturerName}{LecturerName}
\DeclareMathOperator{\Mark}{Mark}

\pagenumbering{gobble}

\title{Домашнее задание №3}
\author{Артем Оганджанян, M3439}
\date{}

\begin{document}

\maketitle

Дано отношение с атрибутами:
$\StudentId$,
$\StudentName$,
$\GroupId$,
$\GroupName$,
$\CourseId$,
$\CourseName$,
$\LecturerId$,
$\LecturerName$,
$\Mark$.

\subsection{\texorpdfstring{Функциональные зависимости}{Task 1}}

\begin{align*}
    \StudentId            &\rightarrow \StudentName  \\
    \GroupId              &\rightarrow \GroupName    \\
    \CourseId             &\rightarrow \CourseName   \\
    \LecturerId           &\rightarrow \LecturerName \\
    \StudentId            &\rightarrow \GroupId      \\
    \StudentId, \CourseId &\rightarrow \Mark, \LecturerId
\end{align*}

\subsection{\texorpdfstring{Ключи}{Task 2}}

Возьмём множество всех атрибутов.
\[
    \StudentId,
    \StudentName,
    \GroupId,
    \GroupName,
    \CourseId,
    \CourseName,
    \LecturerId,
    \LecturerName,
    \Mark
\]
Воспользуемся первыми четырьмя функциональными зависимостями.
\[
    \StudentId,
    \GroupId,
    \CourseId,
    \LecturerId,
    \Mark
\]
Воспользуемся пятой и шестой функциональными зависимостями.
\[
    \StudentId,
    \CourseId
\]
Получили ключ $\{\StudentId, \CourseId\}$. Других ключей нет, т.к. перестановка
порядка использования функциональных зависимостей ничего не меняет.

\subsection{\texorpdfstring{Неприводимое множество функциональных зависимостей}
    {Task3}}
Воспользуемся правилом расщепления.
\begin{align*}
    \StudentId            &\rightarrow \StudentName  \\
    \GroupId              &\rightarrow \GroupName    \\
    \CourseId             &\rightarrow \CourseName   \\
    \LecturerId           &\rightarrow \LecturerName \\
    \StudentId            &\rightarrow \GroupId      \\
    \StudentId, \CourseId &\rightarrow \Mark         \\
    \StudentId, \CourseId &\rightarrow \LecturerId
\end{align*}
В последних двух правилах ни один из атрибутов удалить нельзя.
Ни одно из правил также удалить нельзя. Получаем неприводимое множество
функциональных зависимостей.

\end{document}
